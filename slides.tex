\documentclass{beamer}
\usepackage{amsfonts}
\usepackage[english]{babel}
\usepackage[T1]{fontenc}
\usepackage{graphicx}
\usepackage{hyperref}
\usepackage[utf8]{inputenc}
\usepackage{subfigure}

\usetheme{Berkeley}
\title{Placement constraints for a better QoS in clouds}
\subtitle{Extending BtrPlace to support typing}
\author[]{Mathieu Bivert\\Tutor : Fabien Hermenier}
\institute{Polytech'Nice Sophia}
\date{March $8$, $2013$}

\begin{document}

\begin{frame}{}
\titlepage
\end{frame}

\begin{frame}{Map}
\tableofcontents
\end{frame}

\section{Introduction}
\subsection{Some vocabulary}
\begin{frame}{Some vocabulary}
% description just increase left margin…
\begin{itemize}
	\item{\textbf{Virtual Machine}}, hardware simulated in software
	\item{\textbf{Hypervisor}}, software monitoring a set of virtual
		machines
	\item{\textbf{Node}}, physical server running an hypervisor
	\item{\textbf{Cloud}}, set of physical servers running various
		softwares, providing services to users
	\item{\textbf{QoS}} (Quality Of Service), avaibility of the
		services provided by a Cloud
\end{itemize}
\end{frame}

\subsection{Virtualisation and Cloud}
\begin{frame}{Clouds in business}
Large firms delegates their IT infrastructure to specialized companies
\begin{itemize}
	\item Reduction of the costs (less hardware to buy and manage,
		less software to write, etc.)
	\item Augmentation of the QoS
\end{itemize}
However, by doing so, those firms:
\begin{itemize}
	\item Lose control over their data
	\item Become dependent of another company
\end{itemize}
\end{frame}
\begin{frame}{Different types of services}
\begin{scriptsize}
\begin{tabular}{c|c|c|c}
	\textbf{Acronym} & \textbf{Service} \textbf{provided} & \textbf{Description} & \textbf{Exemple} \\
	\hline
	\hline
	SaaS & Software & Accses to software &
		Gmail\footnote{\url{https://gmail.com}} \\
	\hline
	PaaS & Platform & Software stack for development &
		Google Apps\footnote{\url{https://www.google.com/enterprise/apps/business/}} \\
	\hline
	IaaS & Infrastructure & Access to physical ressources &
		Amazon EC2\footnote{\url{https://aws.amazon.com/ec2/}} \\
	\hline
	DaaS & Data & Access to data&
		Dropbox\footnote{\url{https://www.dropbox.com/}}
\end{tabular}
\end{scriptsize}
\end{frame}
\begin{frame}{Why virtualization in clouds?}
Virtualization may augment QoS by allowing to:
\begin{itemize}
	\item Launch and stop services on the fly
	\item Replicates easily VMs running those services
	\item Facilitate administration
	\item Somehow reduce security issue
\end{itemize}
Also, when QoS is non-critical, it can helps reducing energy
consumption.
\end{frame}
\begin{frame}{How is it done?}
Companies hosting clouds using virtualization pay licence to use
hypervisor. Those licences contains restrictions on the usability
of ressources, such as
\begin{itemize}
	\item CPU, RAM
	\item number of virtual machines
	\item number of NIC and other hardware components
\end{itemize}

Hypervisors come with software which helps manage \textbf{their} VMs.
\end{frame}

\subsection{BtrPlace, a placement manager}
\begin{frame}{Description}
BtrPlace is a software written in Java (GPL) by Fabien Hermenier
(OASIS team). It extensively use the Choco framework (BSD) developped
at the "École des mines de Nantes". Choco gives the programmer tools
to build Java applications modeling
CSP\footnote{Constraint Satisfaction Problem}, or using
CP\footnote{Constraint Programming}.
\end{frame}
\begin{frame}{Goal}
BtrPlace aims to solve the problem of distributing a set of VMs on
a set of nodes efficiently, by following some constraints. The latters
can be :
\begin{itemize}
	\item imposed by the hardware, such as available ressources
	\item given by the user, following his needs (eg. replication
		of VMs)
	\item imposed by hypervisors licences
\end{itemize}

One may think of BtrPlace as a hypervisor's monitor.

\end{frame}
\begin{frame}{BtrPlace and similiar solutions}
Actually, BtrPlace model is unique and is the only one who scales
enough to be usable in real life.

As its competitors, it doesn't support (yet) correctly the typing
of nodes and VMs.

\end{frame}
\section{Adding typing in BtrPlace}
\subsection{More vocabulary}
\begin{frame}{More vocabulary}
\begin{itemize}
	\item{\textbf{Type}}, integer associated to each hypervisor
	\item{\textbf{Deployment}}, operation of rebooting a node and
		eventually changing its hypervisor
	\item{\textbf{Reconfiguration}}, operation during which BtrPlace
		change the placement of VMs on nodes following constraints
	\item{\textbf{Slices}}, finite period of time representing
		ressources usages. Two types of slices, distinguished by
		using the ressources at the
		\begin{itemize}
			\item{\textbf{Consuming slice}} beginning of the
				reconfiguration
			\item{\textbf{Demanding slice}} end of the reconfiguration
		\end{itemize}
\end{itemize}
\end{frame}
\begin{frame}{Proceeding of the work}
We worked incrementally by
\begin{enumerate}
	 \item modeling and implementing a special case of the typing
	 \item modeling and implementing (partially) the general case
	 \item implementing (partially) some constraints
\end{enumerate}
\end{frame}

\subsection{Special case}
\begin{frame}{Model}
\end{frame}
\begin{frame}{Code}
\end{frame}

\subsection{General case}
\begin{frame}{Model}
\end{frame}
\begin{frame}{Code}
\end{frame}

\subsection{Additional constraints}
\begin{frame}{MinPlatform}
\end{frame}
\begin{frame}{MaxVM}
\end{frame}

\begin{frame}{Why type is an integer?}
\begin{itemize}
	\item choco
	\item solves an other problem
\end{itemize}
\end{frame}

\section{Feedback}
\subsection{Problems encountered}
\begin{frame}{Timing management}
\end{frame}

\begin{frame}{Complexity of BtrPlace}
\end{frame}

\subsection{Incomplete work}
\begin{frame}{What's missing?}
\end{frame}

\section{Future plans}
\begin{frame}{Adding documentation to BtrPlace}
\end{frame}

\begin{frame}{Missing models and partial implementation}
\end{frame}

\begin{frame}{Possible usage of the typing}
\end{frame}

\end{document}